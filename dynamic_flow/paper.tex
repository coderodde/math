\documentclass[12pt]{article}

\usepackage[english]{babel}
\usepackage[utf8x]{inputenc}
\usepackage{amsmath}
\usepackage{amssymb}
\usepackage{amsfonts}
\usepackage{scalerel}
\usepackage{empheq}
\usepackage{float}
\usepackage{mathrsfs}
\usepackage{geometry}
\usepackage[colorlinks=true,urlcolor=blue]{hyperref}

\newcommand{\F}{\mathbf{F}}
\newcommand{\FF}{\hat{\Phi}}
\newcommand{\surf}{\mathbf{r}}
\newcommand{\N}{\hat{\mathbf{N}}}
\newcommand{\R}{\mathbb{R}}
\newcommand{\dr}{\partial \surf}
\newcommand{\du}{\partial u}
\newcommand{\dv}{\partial v}
\newcommand{\dt}{\partial t}
\newcommand{\dru}{\frac{\dr}{\du}}
\newcommand{\drv}{\frac{\dr}{\dv}}
\newcommand{\drt}{\frac{\dr}{\dt}}
\newcommand{\dS}{\,\mathrm{d}S}
\newcommand{\ddt}{\,\mathrm{d}t}
\newcommand{\ii}{\mathbf{i}}
\newcommand{\jj}{\mathbf{j}}
\newcommand{\kk}{\mathbf{k}}
\newcommand{\dA}{\, \mathrm{d}A}
\newcommand{\ddu}{\, \mathrm{d}u}
\newcommand{\ddv}{\, \mathrm{d}v}
\newcommand{\dy}{dy}
\newcommand{\dx}{dx}

\title{An example on computing flux across a moving surface in a time-dependent vector field}
\author{Rodion ``rodde'' Efremov}

\begin{document}
 \maketitle

\noindent Suppose we are given a vector field $\F(x, y, z, t), \, \F \colon \R^4 \to \R^3$ describing the motion of fluid at any particular time $t$, and a parametric surface $\mathscr{S}$, $\surf (u, v, t), \surf \colon A \times \R \to \R^3$ that, just like the vector field, evolves with time. The question we want to answer is how to compute the amount of fluid going across the moving surface within a particular time range. First of all we need to define a normal unit vector to the surface $\surf$:
\[
\N = \dru \times \drv \cdot \Big| \dru \times \drv \Big|^{-1}.
\]
Taking motion into account, we have that the surface element at time point $t$ passes the space at rate 
\[
\langle \N , \drt \rangle.
\]
Next we need to consider the actual vector field $\F$ and the flux it generates across an area element. As there is no any dependence between the vector field and the surface, we obtain that the flux across the surface $\mathscr{S}$ at moment $t$ is
\begin{align*}
\Phi(t) &= \iint_{\mathscr{S}} \langle \N, \F \rangle - \langle \N , \drt \rangle \dS \\
		   &= \iint_{\mathscr{S}} \langle \N , \F - \drt \rangle \dS \\
\end{align*}
In order to find out the actual amount of fluid going through a moving surface $\mathscr{S}$, we integrate $\Phi(t)$ over a desired time range $\left[ t_a, t_b \right]$:
\begin{align*}
\FF = \int_{t_a}^{t_b} \Phi(t) \ddt.
\end{align*}

\section*{Example}
Suppose $\F(x, y, z, t) = (0, 0, t)$ and $\surf(u, v, t) = (u, v, 2t)$, $A = \left[ 0, 1 \right]^2$. Now we have
\begin{align*}
\dru &= (1, 0, 0), \\
\drv &= (0, 1, 0), \\
\drt &= (0, 0, 2). \\
\end{align*}
\begin{align*}
\dru \times \drv &= 
\begin{vmatrix}
\ii & \jj & \kk \\
1  & 0   & 0  \\
0  & 1   & 0 \\
\end{vmatrix} \\
      &= \ii \begin{vmatrix}
      0 & 0 \\
      1 & 0 \\
      \end{vmatrix}
      - \jj \begin{vmatrix}
      1 & 0 \\
      0 & 0 \\
      \end{vmatrix}
      + \kk \begin{vmatrix}
      1 & 0 \\
      0 & 1
      \end{vmatrix} \\
      &= \kk \begin{vmatrix}
      1 & 0 \\
      0 & 1 \\
      \end{vmatrix} \\
      &= \kk \\
      &= (0, 0, 1).
\end{align*}
Now $\F(r(u, v, t), t) = (0, 0, t)$, $\drt = (0, 0, 2)$ and $\dru \times \drv = (0, 0, 1)$, so
\begin{align*}
\Phi(t) &= \iint_{\mathscr{S}} \langle \N, \F - \drt \rangle \dS \\
          &= \iint_A \langle \F(r(u,v,t), t) - \drt, \dru \times \drv \rangle \ddu \ddv \\
          &= \iint_A \langle (0, 0, t) - (0, 0, 2), (0, 0, 1) \rangle \ddu \ddv \\
          &= \iint_A \langle (0, 0, t  - 2), (0, 0, 1) \rangle \ddu \ddv \\
          &= \iint_A t - 2 \ddu \ddv.
\end{align*}
Since $A = \left[ 0, 1 \right]^2$,
\begin{align*}
\Phi(t) &= \iint_A t - 2 \ddu \ddv \\
          &= t - 2.
\end{align*}
Now the amount of fluid flowing across the surface $\mathscr{S}$ during time interval $\left[  0, 10 \right]$ is
\begin{align*}
\FF &= \int_0^{10} t - 2 \ddt  \\
      &= \Big[ \frac{1}{2} t^2 - 2t \Big]_{t = 0}^{t = 10} \\
      &= 30.
\end{align*} 
\end{document}