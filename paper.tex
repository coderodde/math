\documentclass[10pt]{article}

\usepackage[english]{babel}
\usepackage[utf8x]{inputenc}
\usepackage{amsmath}
\usepackage{amssymb}
\usepackage{amsfonts}
\usepackage{scalerel}
\usepackage{empheq}
\usepackage{float}
\usepackage{mathrsfs}
\usepackage{geometry}
\usepackage[colorlinks=true,urlcolor=blue]{hyperref}

\title{An example on computing flux through a moving surface in a time-dependent vector field}
\author{Rodion ``rodde'' Efremov}

\begin{document}
 \maketitle

Suppose we are given a vector field $\mathbf{F}(x, y, z, t)$ describing motion of fluid at any particular time $t$, and a parametric curve $\mathbf{r}(u, v, t)$ that, just like the vector field, evolves with time. The question we want to answer is how to compute the amount of fluid going through the surface within a particular time range using a simple example. Let assume that the time interval is $[t_1, t_2]$ and $u \in [0, a], v \in [0, b]$. If
\[
\Phi(t) = \iint\limits_{\mathcal{S}} \mathbf{F} \cdot \mathrm{d}\mathbf{S} 
\]
is the rate of flux at a particular moment, the amount of fluid flowing throught the surface between the moments $t_1$ and $t_2$ is
\[
\int\limits_{t_1}^{t_2} \Phi(t) \, \mathrm{d}t.
\]
Now, let us define $\mathbf{F}(x, y, z, t) = (xt, y^2, x + t)$ and $\mathbf{r}(u, v, t) = (\overbrace{u - t}^x, \overbrace{vt	}^y, \overbrace{uvt}^z).$ Next, we have
\[
\frac{\partial \mathbf{r}}{\partial u} = \frac{\partial}{\partial u} (u - t, vt, uvt) = (1, 0, vt) \qquad \frac{\partial \mathbf{r}}{\partial v} = \frac{\partial}{\partial v}(u - t, vt, uvt) = (0, t, ut),
\]
so
\begin{align*}
\frac{\partial \mathbf{r}}{\partial u} \times \frac{\partial \mathbf{r}}{\partial u} &= 
  \begin{vmatrix}
  \mathbf{i} & \mathbf{j} & \mathbf{k} \\
  1 & 0 & vt \\
  0 & t & ut \\
  \end{vmatrix} \\
  &= 
  \mathbf{i} \begin{vmatrix}
  0 & vt \\
  t & ut \\
  \end{vmatrix}
  -
  \mathbf{j} \begin{vmatrix}
  1 & vt \\
  0 & ut \\
  \end{vmatrix}
  +
  \mathbf{k}
  \begin{vmatrix}
  1 & 0 \\
  0 & t
  \end{vmatrix} \\
  &=
  -vt^2\mathbf{i} -ut\mathbf{j} + t\mathbf{k} \\
  &= (-vt^2, -ut, t).
\end{align*}
Also 
\[
\mathbf{F} = (xt, y^2, z + t) = (ut - t^2, v^2t^2, uvt + t),
\]
which leads us to 
\begin{align*}
\Phi(t) &= \iint\limits_{\mathcal{S}} \mathbf{F} \cdot \mathrm{d}\mathbf{S} \\
          &= \int\limits_0^b \int\limits_0^a (ut - t^2, v^2t^2, uvt + t) \cdot (-vt^2, -ut, t) \; \mathrm{d}u \; \mathrm{d}v \\
          &= \int\limits_0^b \int\limits_0^a vt^4 - uvt^3 - uv^2t^3 + uvt^2 + t^2 \; \mathrm{d}u \; \mathrm{d}v \\
          &= \int\limits_0^b  \Bigg[ uvt^4 - \frac{1}{2} u^2vt^3 - \frac{1}{2} u^2v^2t^3 + \frac{1}{2} u^2vt^2 + ut^2 \Bigg]_{u = 0}^{u = a}\; \mathrm{d}v \\
          &= \int\limits_0^b avt^4 - \frac{1}{2}a^2vt^3 - \frac{1}{2}a^2v^2t^3 + \frac{1}{2}a^2vt^2 + at^2 \; \mathrm{d}v \\
          &= \Bigg[ \frac{1}{2}av^2t^4 - \frac{1}{4}a^2v^2t^3 - \frac{1}{6}a^2v^3t^3 + \frac{1}{4}a^2v^2t^2 + avt^2 \Bigg]_{v = 0}^{v = b} \\
          &= \frac{1}{2}ab^2t^4 - \frac{1}{4}a^2b^2t^3 - \frac{1}{6}a^2b^3t^3 + \frac{1}{4}a^2b^2t^2 + abt^2 \\
          &= \frac{ab^2}{2} t^4 - \Bigg(\frac{a^2b^3}{6} + \frac{a^2b^2}{4}\Bigg)t^3 + \Bigg( \frac{a^2b^2}{4} + ab\Bigg)t^2.
\end{align*}
Now let us assign $a = 1$ and $b = 3$; we obtain
\begin{align*}
\Phi(t) &= \frac{9}{2}t^4 - \Bigg( \frac{27}{6} + \frac{9}{4}\Bigg)t^3 + \Bigg( \frac{9}{4} + 3\Bigg)t^2 \\
          &= \frac{9t^4}{2}  - \frac{27t^3}{4} + \frac{21t^2}{4}.
\end{align*}
Let us choose $t_1 = 1$ and $t_2 = 3$; the total amount of fluid flowed through the surface is
\begin{align*}
\int\limits_{t_1}^{t_2} \Phi(t) \; \mathrm{d} t &= \int\limits_1^3 \frac{9t^4}{2}  - \frac{27t^3}{4} + \frac{21t^2}{4} \; \mathrm{d}t \\
    &= \Bigg[ \frac{9}{10} t^5 - \frac{27}{16} t^4 + \frac{21}{12} t^3 \Bigg]_{t = 1}^{t = 3} \\
    &= \Bigg[ \frac{2187}{10} - \frac{2187}{16} + \frac{567}{12} \Bigg] - \Bigg[ \frac{9}{10} - \frac{27}{16} + \frac{21}{12} \Bigg] \\
    &= \frac{2178}{10} - \frac{2160}{16} + \frac{546}{12} \\
    &= 128.3.
\end{align*}
\end{document}